\documentclass[letterpaper,11pt]{article}

%\newtheorem{definition}{Definition}
%\newtheorem{theorem}{Theorem}
%\newtheorem{lemma}[theorem]{Lemma}
%\newtheorem{proposition}[theorem]{Proposition}
%\newtheorem{corollary}[theorem]{Corollary}

\def\Proof{{\bf Proof.}}

\newcommand{\update}{U}


\newcommand{\IsSat}[1]{\mathit{IsSat}(#1)}
\newcommand{\IsValid}[1]{\mathit{Is\!Valid}(#1)}


\newcommand{\loris}{\textcolor[rgb]{0.00,0.00,1.00}}

\begin{document}

\title{Symbolic counter machines}

\author{...}
\maketitle

\section{Definition}

\begin{definition}
A (nondeterministic) symbolic counter machine is a tuple $A=(Q,q_0,F,C,\delta)$,
where
\begin{itemize}
\item $Q$ is a finite set of states,
\item $q_0\in Q$ is an initial state,
\item $F\subseteq Q$ is a set of accepting states,
\item $C$ is a finite set of counters,
\item $\delta$ is a transition relation containing tuples of the form
\begin{description}
\item[Test and set:] $(q,C_t,\varphi,\rho,q')$ such that:
\begin{itemize}
\item $q,q'\in Q$ are the source and target states respectively;
\item $C_t\subseteq$ is the subset of counters being tested;
\item $\varphi \subseteq lin(C_t)$ is a linear guard over the counters $C_t$; and
\item $\rho\subseteq lin(C_t)$ is a linear counter update that nondeterministically maps the counters $C_t$ to a value satisfying $\rho$.
\end{itemize}

\item[Decrement:] $(q,c,q')$ such that $q,q'\in Q$ are the source and target states respectively, and $c\in C$ is a counter being decremented;

\end{description}
\end{itemize}
\end{definition}


Intuitively the machine $A$ starts in state $q_0$ with all the counters set to $0$.
Each test and set transition $(q,C_t,\varphi,\rho,q')$ checks whether the set of counters $C_t$ have
values satisfying
$\varphi$ and, if this is the case, it updates the values of the 
counters in $C_t$ to a value satisfying $\rho$.
Each decrement transition $(q,c,q')$ subtracts $1$ from the value of the counter $c$.
A run reaching the final state is a successful run.

\loris{TODO: define semantics formally}

\section{Reachability algorithm}
The reachability problem is to find whether, given two states $q_1$ and $q_2$ in $Q$ there exists a path from $q_1$ to $q_2$.
We solve this problem via a Kleene-style reachability algorithm.

Let $S(q_1,q_2,R,V)$ be the set of counter values we can obtain when starting in state $q_1$ (with counter values 0) and reaching state $q_2$ such that:
\begin{itemize}
\item a path can use ONLY the test-and-set transitions in $R$ but all decrement transitions;
\item each counter $c\in V$ must be reset at least once (in some sense $V\subseteq V(R)$), and no other counter can be reset.
\end{itemize}
We can define this quantity as follows.
\[
\begin{array}{rcl}
S(q_1,q_2,R, \emptyset) & = &P\text{ the semilinear set induced by all}\\
&& \text{decrement transitions from }q_1\text{ to  }q_2\\
S(q_1,q_2,R,V) & = &\emptyset\text{ if } \neg V\subseteq V(R)\\
&&\\
S(q_1,q_2,R,V) & = & \bigcup\limits_{r=(q_3,V_r, \varphi, \rho, q_4)\in R} S(q_1,q_2, R\setminus\{r\},V)\\
&&\\
& \cup & \bigcup\limits_{V_1\cup V_2 \cup V_r=V} S(q_1,q_3, R\setminus \{r\}, V_1) \proj \overline{V_2\cup V_r}\\
&         &  \quad\quad\quad \quad\quad\quad + S(q_4,q_2,R\setminus\{r\}, V_2)\\
&&\\
 & \cup  & \bigcup\limits_{V_1\cup V_2\cup V_3 \cup V_r=V} S(q_1,q_3, R\setminus\{r\},V_1)\proj \overline{V_2\cup V_3\cup V_r}+\\
&         &  \quad\quad\quad\quad\quad\quad\quad\quad\quad\quad\quad\quad + \gamma(R\setminus \{r\},r,V_2)\proj \overline{V_3\cup V_r} \\
&         &  \quad\quad\quad\quad\quad\quad\quad\quad\quad\quad\quad\quad\quad\quad  + S(q_4,q_2, R\setminus \{r\},V_3)\\
\end{array}
\]
The interesting case is the third rule. 
For each $r\in R$ we take into account three possibilities:
\begin{itemize}
\item the paths that do not use $r$ but still reset all the variables in $V$;
\item the paths that use $r$ exactly once. In this case we iterate over all possible sets $V_1$ and $V_2$ for which the union is $V$ and assume that
		all the  that th
\item the paths that use $r$ multiple times. This is the most interesting case in which we need to use the star operation. This is done via the function $\gamma$
		that takes the star of all the possible combinations of $R$.
\end{itemize}
\section{Complement}

\cite{jurg14}

\bibliographystyle{alpha}
\bibliography{dmatheory}

\end{document}

