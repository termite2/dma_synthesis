\documentclass{article}

\usepackage{amsfonts,stmaryrd}
\usepackage{amsmath}
\usepackage{amssymb}
\usepackage{color}
\usepackage{mathrsfs}

\usepackage{xspace}

%\newtheorem{definition}{Definition}
%\newtheorem{theorem}{Theorem}
%\newtheorem{lemma}[theorem]{Lemma}
%\newtheorem{proposition}[theorem]{Proposition}
%\newtheorem{corollary}[theorem]{Corollary}

\def\Proof{{\bf Proof.}}

\newcommand{\update}{U}


\newcommand{\IsSat}[1]{\mathit{IsSat}(#1)}
\newcommand{\IsValid}[1]{\mathit{Is\!Valid}(#1)}


\newcommand{\loris}{\textcolor[rgb]{0.00,0.00,1.00}}

\begin{document}

\title{Symbolic counter machine}

\author{}
\maketitle

\section{Definition}

\begin{definition}
A (nondeterministic) symbolic counter machine is a tuple $A=(Q,q_0,F,C,S,\delta)$, 
where
\begin{itemize}
\item $Q$ is a finite set of states,
\item $q_0\in Q$ is an initial state,
\item $F\subseteq Q$ is a set of accepting states,
\item $C$ is a finite set of counters,
\item $S$ is a finite set of symbolic values,
\item $\delta$ is a transition relation containing tuples of the form
\begin{description}
\item[Test and set:] $(q,\varphi,\update,q')$ such that:
\begin{itemize}
\item $q,q'\in Q$ are the source and target states respectively;
\item $\verphi\subseteq C$ is the activation guard; and
\item $\update: C \mapsto S$ is an counter update that maps a subset of the counters $C$ so values in $S$.
\end{itemize}

\item[Decrement:] $(q,c,q')$ such that $q,q'\in Q$ are the source and target states respectively, and $c\in C$ is a counter being decremented;
    
\end{description}
\end{itemize}
\end{definition}


Intuitively the machine $A$ starts in state $q_0$ with all the counters set to $0$. 
Each test and set transition $(q,\varphi,\update,q')$ checks whether the set of counters
$\varphi$ has value $0$ and, if this is the case, it updates the values each
counter $c$ for which $\update$ is defined to $\update(c)$.
Each decrement transition $(q,c,q')$ subtracts $1$ from the value of the counter $c$.
A run reaching the final state is a successful run.

\loris{TODO: define semantics formally}

\section{Reachability algorithm}

Depending on the values assigned to the symbolic values in $S$, final states can be reachable or not.

The reachability problem is to find whether there exist values of $S$ for which $A$ can reach the final states.

\section{Complement}


\end{document}

