\documentclass{article}

\usepackage[final,formats]{listings}
\usepackage{times}
\usepackage{graphicx}
\usepackage{cite}
%\titlespacing*{\subsection}{0pt}{1.1\baselineskip}{\baselineskip}

\newcommand{\comment}[1]{\textbf{\textsl{#1}}}
%\newcommand{\comment}[1]{}
\newcommand{\leonid}[1]{\comment{[leonid]#1}}

\newcommand{\src}[1]{\texttt{#1}}
\newcommand{\cir}[1]{\raisebox{.5pt}{\textcircled{\raisebox{-.9pt} 
{#1}}}}

\newcommand{\mypara}[1]{\vspace{1mm}\noindent\emph{#1}~~}

\setlength{\textfloatsep}{8pt plus 1.0pt minus 2.0pt}
\renewcommand{\topfraction}{0.9}
\renewcommand{\textfraction}{0.1}

\renewcommand{\ttdefault}{pcr}

\lstnewenvironment{tsllisting}[1][]
{\lstset{
    escapeinside={(*@}{@*)},
    basicstyle=\ttfamily\scriptsize,
    keywordstyle=\bfseries,
    keywordstyle=\bfseries,
    sensitive=false,
    morekeywords={pause, if, else, typedef, task, export, template, endtemplate, process, controllable, forever, wait, return, assert, goal, instance},
    identifierstyle=, 
    commentstyle=\slshape, 
    stringstyle=, showstringspaces=false,
    sensitive=false,
    morecomment=[s]{/*}{*/},
    morecomment=[l]{//},
    numberstyle=\tiny,
    stepnumber=1,
    numbersep=1pt,
    emphstyle=\bfseries,
    belowskip=0pt,
    aboveskip=0pt,
    #1
}}{}

\sloppy

\begin{document}

\pagestyle{headings}  % switches on printing of running heads

\title{Fictional Ethernet Controller Case Study}

\author{Leonid}

\maketitle

\section{Overview}

The example models the send queue of a fictional network controller.  The queue 
is organized as a cicrular buffer where every entry is a descriptor that stores 
address and size of a packet fragment along with two flags: the \texttt{last} 
flag is set to true iff this is the last fragment of a packet.  The 
\texttt{own} flag is equal to one of two constants: \texttt{DEV} and 
\texttt{OS}.  The former means that the descriptor is ready to be read by the 
device.  The latter means that the descriptor is owned by the OS and cannot yet 
be accesses by the device.  Two \texttt{START} and \texttt{END} pointers store
the offset of the first descriptor in the queue and the descriptor following 
the last used descriptor.  We are going to assume that the buffer is always 
parsed starting from the \texttt{START} pointer, so we are going to ignore it.

We consider the \texttt{enqueue} operation that adds a packet to the queue.  
The new packet is represented by a list of fragments, where every fragment 
consists of address and size fields.

Figure~\ref{f:arch} shows the various data types and transducers over these 
types used to formalize this example.  The data types are:

\begin{figure}[t]
    \center
    \includegraphics[width=0.9\linewidth]{imgs/arch.pdf}
    \caption{}\label{f:arch}
\end{figure}


\begin{itemize}
    \item The ``circular buffer'' type represents the actual 
        physical layout of the device transmit queue
    \item The ``new packet'' type models a packet to be added to 
        the queue.  This type is defined by the OS.
    \item The ``abstract queue'' type describes the abstract view 
        of the transmit queue
\end{itemize}

The transducers are:
\begin{itemize}
    \item The concrete append transducer takes a circular buffer 
        and a new packet and output the buffer with the packet 
        appended to it
    \item The abstractor transducer takes a concrete buffer and 
        generated its abstract representation
    \item The abstract append transducer takes an abstract queue 
        and a new packet and outputs a modified abstract queue 
        with the packet added to it
\end{itemize}

The equivalence checking problem is, given a specification of all 
transducers, verify that for any valid combination of inputs (see 
definition below), the two outputs produced by the transducers in 
the diagram are identical.

\section{Type definitions}

Figure~\ref{f:types} illustrates the tree data types used in the 
example.  Table~\ref{t:alphabet} defines their alphabets.  

\begin{figure}[t]
    \center
    \includegraphics[width=\linewidth]{imgs/types.pdf}
    \caption{}\label{f:types}
\end{figure}

\begin{table}
    \begin{tabular}{|l|p{0.7\linewidth}|}
    \end{tabular}
    \label{t:alphabet}
\end{table}

For each data type, we define a \emph{safety automaton} that 
accepts exactly the set of valid words of this type.  Safety 
automata play a dual role in verification.  First, they constrain 
the set of inputs that must be handled by the transducers (and on 
which the transducers must be equivalent).  Second, they constrain 
the set of valid output strings generated by each transducer to 
well-formed strings of the appropriate type.

\subsection{The abstract queue safety automaton}

The safety automaton for the abstraft queue type in 
Figure~\ref{f:abs_safety} only requires that the \texttt{EOI} 
symbol appears last in the word.

\begin{figure}[t]
    \center
    \includegraphics[width=0.3\linewidth]{imgs/abs_safety.pdf}
    \caption{The abstract queue safety automaton}\label{f:abs_safety}
\end{figure}


\subsection{The new packet safety automaton}

The safety automaton for the new packet type 
(Figure~\ref{pkt_safety}) uses two registers: \texttt{nfrags}, 
which stores the number of fragments in the packet, and 
\texttt{cnt}, which counts the number of fragments parsed by the 
automaton.  The automaton accepts words that start with a 
\texttt{NEWPKT} header and consist of exactly the number of 
fragments specified in the header. 

\begin{figure}[t]
    \center
    \includegraphics[width=0.7\linewidth]{imgs/pkt_safety.pdf}
    \caption{The new packet safety automaton}\label{f:pkt_safety}
\end{figure}




\subsection{The circular buffer safety 
automaton}\label{s:conc_safety}

The safety automaton for the circular buffer type 
(Figure~\ref{conc_safety}) uses the \texttt{is\_last} boolean 
variable to keep track of whether the automaton is parsing the 
last descriptor of a packet (note that, although \texttt{is\_last} 
is modeled as a register, it could also be simply integrated into 
the state space of the automaton).  The automaton requires that 
the \texttt{END} symbol appears after the last fragment of a 
packet (transition 1-3), that it is followed by at least one 
fragment owned by the OS (3-4) and that all subsequent fragments 
are owned by the OS (4-4).  

\begin{figure}[t]
    \center
    \includegraphics[width=0.7\linewidth]{imgs/conc_safety.pdf}
    \caption{The circular buffer safety 
    automaton}\label{f:conc_safety}
\end{figure}



\section{Transducers}

We specify the various transformations as symbolic transducers 
with the following non-standard features:

\begin{itemize}
    \item A transducer can take multiple input words (up to two in 
        this example).  At every transition, the transducer picks 
        the word to read the next symbol from depending on the 
        current state.  

    \item A restricted form of registers.
        \begin{itemize}
            \item The register is initialized to one of input 
                values
            \item The register is only used in comparisons against 
                counters (see below)
        \end{itemize}

    \item Counters.  A counter is another special kind of register 
        that is only accessed in the following ways:
        \begin{itemize}
            \item Resetting the counter to 0
            \item Incrementing by 1
            \item Comparing against a threshold value.  The 
                threshold is an expression over other registers.  
                In the circular buffer example the threshold does 
                not change between two resets.
        \end{itemize}
\end{itemize}

\subsection{The abstractor transducer}

The abstractor automaton in Figure~\ref{f:abstractor} transforms a 
circular buffer into an abstract request queue.  It merges 
multiple descriptors into a single packet.  The \texttt{is\_last} 
boolean variable is true iff the transducer is currently parsing 
or has finished parsing the last fragment of a packet.  In state 1 
it accepts either a descriptor owned by the device or the 
\texttt{END} symbol indicating the end of the queue.  If the 
transducer reads a descriptor and \texttt{is\_last} is true, then 
it outputs a \texttt{PKT} symbol, which indicates the start of a 
new packet.  Otherwise, the content of the fragment parsed in 
state 2 is added to the current packet.  Upon reading the 
\texttt{END} symbol, the transducer outputs \texttt{EOI} and skips 
the rest of the input word (state 5).

\begin{figure}[t]
    \center
    \includegraphics[width=0.8\linewidth]{imgs/abstractor.pdf}
    \caption{The abstractor transducer}\label{f:abstractor}
\end{figure}


\subsection{The abstract append transducer}
 
This transducer (Figure~\ref{f:abs_append}) takes two inputs: (1) 
an abstract packet queue and (2) a new packet.  Transitions of the 
automaton are prefixed with ``1'' or ``2'', which indicate the 
input read by the transition.

\begin{figure}[t]
    \center
    \includegraphics[width=0.7\linewidth]{imgs/abs_append.pdf}
    \caption{The abstract append transducer}\label{f:abs_append}
\end{figure}


The transducer first copies the entire first word (sans the EOI 
symbol) to the output (states 1 and 2) and then copies the content 
of all fragments of the new packet to the output (states 3-4), 
prefixing them by the \texttt{PKT} symbol.  Upon reaching the end 
of the second input, it output \texttt{EOD} and \texttt{EOI} 
symbols (transition 3-5).

\subsection{The concrete append transducer}

This transducer (Figure~\ref{f:conc_append}) models the 
implementation of the append function over the circular buffer.  
It takes the circular buffer as its first input and the new packet 
as the second input.  The \texttt{nfrags} register and the 
\texttt{cnt} counter have the same meaning as in the circular 
buffer safety automaton (Section~\ref{s:conc_safety}).

\begin{figure}[t]
    \center
    \includegraphics[width=\linewidth]{imgs/conc_append.pdf}
    \caption{The concrete append transducer}\label{f:conc_append}
\end{figure}

The transducer first copies all circular buffer entries up to the 
\texttt{END} marker to the output (states 1 and 2).  States 3 and 
4 of the automaton copy a fragment of the new packet to the next 
circular buffer descriptor.  This corresponds to overwriting an 
existing descriptor; hence the automaton must remove this 
descriptor and all its content from the first input word.  This is 
accomplished in states 5 and 6.  After iterating through all 
fragments of the new packet (transition 6-7), the transducer reads 
the \texttt{EOI} symbol from the second input word.  In state 8, 
it iterates through the rest of the first input word, which 
corresponds to the remaining unmodified entries of the circular 
buffers and copies them to the output.  

\section{Next modelling steps}

This simple case study is missing a number of features needed to 
model real-world DMA operation.  In particular, we need to:

\begin{itemize}
    \item model the fact that the circular buffer is bounded
    \item model the second ``shadow'' circular buffer that stores 
        virtual addresses of DMA buffers
    \item model packet fragmentation/defragmentation operations
    \item model more comples descriptor formats
    \item find a way to specify concurrent modifications of the 
        circular buffer by the device
    \item model other DMA operations, such as initializing and 
        clearing of the buffer

\end{itemize}

\end{document}
